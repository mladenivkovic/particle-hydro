\newpage
%==================================================================================
\section{Searching For Neighbouring Particles}\label{chap:neighboursearch}
%==================================================================================



Hydrodynamics is local, and the particle interactions will only depend on a few neighbouring particles.
How many neighbours we use is a parameter that we can set as \texttt{nngb} in the parameter file.
So for every particle, we only want them to interact with a given number of neighbours, which we need to identify first.
A naive way of doing that is comparing for each particle the distances of all other particles.
However, that results in an $\mathcal{O}(N^2)$ algorithm, which can get quite expensive when the number of particles increases, especially since at least in principle the neighbour search needs to be done every time step.
Instead, a more efficient way divide up the simulation box in cells, and distribute particles based on their position in those cells.









%==================================================================================
\subsection{The Cell Grid}
%==================================================================================

There are many good ways of distributing particles in cells, for example we could build adaptive trees and ensure a minimal or maximal number of particles in each cell.
But that is not the point of this program, where we want to focus on the hydrodynamics.
So instead, I built a simple uniform grid, where all cells have equal size.

The determination of the smoothing length (see Section \ref{chap:smoothing_length}) needs to be done iteratively.
Hence we need enough particles so that the iteration can be performed.
The criterion for ``enough particles'' is set to be that the number of particles in any given cell and all its neighbours must be at least \texttt{CELL_MIN_PARTS_IN_NEIGHBOURHOOD_FACT} $\times$ \texttt{nngb} particles.
\texttt{CELL_MIN_PARTS_IN_NEIGHBOURHOOD_FACT} is set in \texttt{defines.h}.
If this is not the case, we reduce the number of cells, making each cell bigger, and try again.







%==================================================================================
\subsection{Finding Neighbours}
%==================================================================================


