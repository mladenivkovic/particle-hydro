\newpage
%==================================================================================
\section{Searching For Neighbouring Particles}\label{chap:neighboursearch}
%==================================================================================



Hydrodynamics is local, and the particle interactions will only depend on a few neighbouring particles.
How many neighbours we use is a parameter that we can set as \texttt{nngb} in the parameter file.
So for every particle, we only want them to interact with a given number of neighbours, which we need to identify first.
A naive way of doing that is comparing for each particle the distances of all other particles.
However, that results in an $\mathcal{O}(N^2)$ algorithm, which can get quite expensive when the number of particles increases, especially since at least in principle the neighbour search needs to be done every time step.
Instead, a more efficient way divide up the simulation box in cells, and distribute particles based on their position in those cells.









%==================================================================================
\subsection{The Cell Grid}
%==================================================================================

There are many good ways of distributing particles in cells, for example we could build adaptive trees and ensure a minimal or maximal number of particles in each cell.
But that is not the point of this program, where we want to focus on the hydrodynamics.
So instead, I built a simple uniform grid, where all cells have equal size.

The determination of the smoothing length (see Section \ref{chap:smoothing_length}) needs to be done iteratively.
Hence we need enough particles so that the iteration can be performed.
The criterion for ``enough particles'' is set to be that the number of particles in any given cell and all its neighbours must be at least \texttt{CELL_MIN_PARTS_IN_NEIGHBOURHOOD_FACT} $\times$ \texttt{nngb} particles.
\texttt{CELL_MIN_PARTS_IN_NEIGHBOURHOOD_FACT} is set in \texttt{defines.h}.
If this is not the case, we reduce the number of cells, making each cell bigger, and try again.







%==================================================================================
\subsection{Finding Neighbours to interact with}
%==================================================================================

Once the smoothing length $h_i$ (see Section \ref{chap:smoothing_length} on how that's done) for a particle $i$ is set, we don't want to keep all the particles from all surrounding cells in memory as neighbours, but only keep those that are going to interact with this particle, i.e. only those that fit within the compact support radius $H_i$ of the kernel in use.
To this end, we store for each particle $i$ the neighbouring particle indices with distance smaller than $H_i$ in the \verb|int* neigh_iact|, where the index refers to the particle's index in the \texttt{particles} array, as well as the distance $r_{ij}$ to each neighbouring particle $j$ into the array \verb|float* r|. 
These arrays will be sorted by ascending particle index to make interactions during particle loops simpler (see Section \ref{chap:particle_loops}).


