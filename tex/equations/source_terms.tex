\newpage
%===============================================================================
\section{Source Terms: Dealing with External Forces}\label{chap:sources}
%===============================================================================



\todo{Needs revision}



So far, the hydro solvers have been dealing with the homogeneous, i.e. source free, Euler equations.
But what if external forces like gravity or combustion are present?
Then the full governing equations are given by

\begin{align}
	\deldt
	\begin{pmatrix}
		\rho \\
		\rho \V \\
		E
	\end{pmatrix}
	+
	\nabla \cdot
	\begin{pmatrix}
		\rho \V\\
		\rho \V \otimes \V + p\\
		(E + p) \V
	\end{pmatrix}
	=
	\begin{pmatrix}
		0\\
		\rho \mathbf{a}\\
		\rho \mathbf{a} \V
	\end{pmatrix}
\end{align}





where $\mathbf{a}$ is some external (velocity-independent) acceleration resulting from an external force.
In 1D, the equations can be written as


\begin{align}
	\deldt
	\begin{pmatrix}
		\rho \\
		\rho u \\
		E
	\end{pmatrix}
	+
	\deldx
	\begin{pmatrix}
		\rho u\\
		\rho u^2 + p\\
		(E + p) u
	\end{pmatrix}
	&=
	\begin{pmatrix}
		0\\
		\rho a\\
		\rho a u
	\end{pmatrix}
	\\
%
	\deldt \U + \deldx \F(\U) &= \mathbf{S}(\U)
\end{align}





Just like for the dimensional splitting approach, one can derive first and second order accurate schemes to include source terms by using the operator splitting technique:
We split the PDE and solve for the two ``operators'' $\deldx \F(\U)$ and $\mathbf{S}(\U)$ individually, where we use the updated solution of the application of the first operator as the initial conditions for the application of the second operator.

Let $F(\Delta t)$ describe the operator $\deldx \F(\U)$ applied over a time step $\Delta t$, and let $S(\Delta t)$ be the operator $\mathbf{S}(\U)$ over a time step $\Delta t$, such that

\begin{align}
	\DELDT \U + F [\U] = S [\U]
\end{align}

analytically.
To obtain a first order accurate scheme, we need to complete the following two steps:

\begin{enumerate}


\item 	We obtain an intermediate result $\U^{n+\half}$ by solving the homogeneous Euler equation over the full time interval $\Delta t$ :
	\begin{align}
		&\begin{cases}
			\text{PDE: } & \deldt{\U} + \deldx{\F(\U)} = 0\\
			\text{IC: } &  \U^n\\
		\end{cases}\\
		&\U^{n+\half} = F(\Delta t)[\U^n]
	\end{align}

\item  then we evolve the solution to the final $\U^{n+1}$ by solving source ODE over the full time interval $\Delta t$ :

	\begin{align}
		&\begin{cases}
			\text{PDE: } & \deldt{\U}  = \mathbf{S}\\
			\text{IC: } & \U^{n+\half} \\
		\end{cases}\\
		& \U^{n+1} = S(\Delta t)[\U^{n+\half}]
	\end{align}
	
	Possible integrators to solve this equation is discussed in section \ref{chap:integrators}.

\end{enumerate}



The order of which operator is solved first doesn't matter.
The method is first order accurate as long as the operator $F$ is also at least first order accurate.
It can be shown that the operator splitting technique is second order accurate if we instead evolve the source term over half a time step twice:

\begin{equation}
	\U^{n+1} = S(\Delta t / 2)  F(\Delta t) S(\Delta t / 2) [\U^n]
\end{equation}

To achieve second order accuracy, the operator $F$ needs also to be at least second order accurate.



Finally, we already use the operator splitting technique to get a multidimensional scheme.
How will a second order accurate scheme in multiple dimensions look like?

Let us express the two dimensional Euler equations as






\begin{align}
	\deldt{ 
		\begin{pmatrix}
			\rho \\ \rho u \\ \rho v \\ E
		\end{pmatrix}
		}
	+ \deldx {
		\begin{pmatrix}
			\rho u\\
			\rho u^2  + p\\
			\rho uv\\
			(E + p) u
		\end{pmatrix}
	} 
	+ \frac{\del}{\del y}
		\begin{pmatrix}
			\rho v\\
			\rho uv\\
			\rho v^2  + p\\
			(E + p) v
		\end{pmatrix}
	&= 	\begin{pmatrix}
			0\\
			\rho a_x\\
			\rho a_y\\
			\rho \mathbf{a} \cdot \V
		\end{pmatrix}\\
%
    \DELDT{ \U } + \DELDX{\F(\U)} + \frac{\del \mathbf{G}(\U)}{\del y} &= \mathbf{S}(\U)
\end{align}


Let again $F(\Delta t)$ describe the operator $\deldx \F(\U)$ applied over a time step $\Delta t$, $G(\Delta t)$ describe the operator $\frac{\del}{\del y} \mathbf{G}(\U)$ applied over a time step $\Delta t$, and let $S(\Delta t)$ be the operator $\mathbf{S}(\U)$ over a time step $\Delta t$, such that

\begin{align}
	\DELDT \U + F[\U] + G[\U] = S [\U]
\end{align}

analytically.
Let $H[\U]$ be the operator describing the homogeneous part of the Euler equations, i.e. $H[\U] = F[\U] + G[\U]$.
The condition for first order accuracy is that in the scheme

\begin{align}
	\U^{n+1} = H(\Delta t) S(\Delta t) [\U^n]
\end{align}

$H[\U]$ must be at least first order accurate.
This is given when we split the $F$ and $G$ operators just like for dimensional splitting:

\begin{align}
	H(\Delta t) [\U] = F(\Delta t) G(\Delta t) [\U]
\end{align}


So finally, a first order accurate two dimensional method is given by

\begin{align}
	\U^{n+1} = F(\Delta t) G(\Delta t) S(\Delta t) [\U]
\end{align}


The same logic can be applied for second order accuracy.
A second order accurate operator $H(\Delta t)$ is given by

\begin{align}
	H(\Delta t) [\U] = F(\Delta t / 2) G(\Delta t) F(\Delta t / 2)[\U]
\end{align}


So finally, a second order accurate two dimensional method is given by

\begin{align}
	\U^{n+1} &= S(\Delta t / 2) H(\Delta t) S(\Delta t / 2)[\U]\\
			 &= S(\Delta t / 2) F(\Delta t / 2) G(\Delta t) F(\Delta t / 2) S(\Delta t / 2)[\U]
\end{align}











%===============================================================================
\subsection{Implemented Sources}\label{chap:implemented-sources}
%===============================================================================




\subsubsection*{Constant Acceleration}

Probably the simplest external force terms are constant accelerations, i.e. the acceleration $\mathbf{a}$ doesn't change during the entire simulation.
The value of the acceleration term $\mathbf{a}$ in each (Cartesian) direction can be set in the parameter file.









\subsubsection*{Constant Radial Acceleration}

This one was mainly implemented to see whether acceleration works at all, and whether it works correctly.
The acceleration is assumed to be constant and radial with origin at the middle of the simulation box, $(0.5, 0.5)$ in 2D or $(0.5)$ in 1D.
A positive acceleration value points away from the origin at the center of the box, a negative towards it.












%===============================================================================
\subsection{Implemented Integrators}\label{chap:integrators}
%===============================================================================

\newcommand{\KB}{\mathbf{K}}
\newcommand{\SB}{\mathbf{S}}

Throughout, we assume that the source terms $\SB$ may depend both on the time $t$ and on the conserved state $\U$: $\SB = \SB(t, \U)$


\subsubsection*{Runge Kutta 2}


The Runge Kutta 2 second-order explicit integrator is given by
\begin{align*}
	\KB_1 &= \Delta t \ \SB(t^n, \U^n)\\
	\KB_2 &= \Delta t \ \SB(t^n + \Delta t, \U^n + \KB_1)\\
	\U^{n+1} &= \U^n + \frac{1}{2} (\KB_1 + \KB_2)
\end{align*}




\subsubsection*{Runge Kutta 4}


The Runge Kutta 4 fourth-order explicit integrator is given by
\begin{align*}
	\KB_1 &= \Delta t \ \SB(t^n, \U^n)\\
	\KB_2 &= \Delta t \ \SB(t^n + \frac{1}{2}\Delta t, \U^n + \frac{1}{2}\KB_1)\\
	\KB_3 &= \Delta t \ \SB(t^n + \frac{1}{2}\Delta t, \U^n + \frac{1}{2}\KB_2)\\
	\KB_4 &= \Delta t \ \SB(t^n + \Delta t, \U^n + \KB_3)\\
	\U^{n+1} &= \U^n + \frac{1}{6} (\KB_1 + 2\KB_2 + 2\KB_3 + \KB_4)
\end{align*}



