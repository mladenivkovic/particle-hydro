\newpage
%==================================================================================
\section{Smoothing Lengths}\label{chap:smoothing_length}
%==================================================================================


%==================================================================================
\subsection{Computing the Smoothing Lengths}
%==================================================================================



There is no unique way of defining how the smoothing length should be set.
You could have for example a fixed smoothing length for all particles everywhere, or let it evolve over time based on the densest region in your simulation, or whatnot.
However, it is desirable to resolve both clustered and sparse regions of your simulation evenly, i.e. with a roughly constant ratio of $h$ to the mean local particle separation.
Thus a natural choice for setting the smoothing length is to relate to the local number density of particles, i.e.
\begin{equation}
	h(\R_i) \propto n(\R_i) ^ {-1/\nu}
\end{equation}
for $\nu$ dimensions.
Let $\eta$ be a parameter specifying the smoothing length in units of the mean particle spacing $n ^ {-1/\nu} = (\rho/m) ^ {-1/\nu}$.
Then
\begin{equation}
	h(\R_i) = \eta n(\R_i) ^ {-1/\nu} = \eta \left(\frac{\rho_i}{m_i}\right) ^ {-1/\nu} = \eta \left(\frac{m_i}{\rho_i}\right) ^ {1/\nu}
\end{equation}
and simultaneously, we have
\begin{equation}
	\rho_i = \sum_{n} m_j W(|\R_i - \R_j|, h_i)
\end{equation}
to solve, where $n$ refers to all neighbouring particles of $i$, including itself.

This set of equations can be solved iteratively using the Newton-Raphson root finding algorithm:
For a differentiable function $f(x)$, we find $x_{root}$ that satisfies $f(x_{root}) = 0$ by iterating
\begin{equation}
	x_{n+1} = x_{n} - \frac{f(x_n)}{\DDX{f}(x_n)}
\end{equation}
until we reach some tolerance $\varepsilon$:
\begin{equation}
	|f(x_n)| \leq \varepsilon
\end{equation}


In our scenario, we have
\begin{align*}
	h_i &= \eta \left(\frac{m_i}{\rho_i}\right) ^ {1/\nu} \\
	\Rightarrow f 
		&= f(h_i) = h_i - \eta \left(\frac{m_i}{\rho_i}\right) ^ {1/\nu} = 0\\
\end{align*}
and we can use the root finding algorithm.
All we need is $\frac{\de f}{\de h_i}$ :
\begin{align}
	\frac{\de f}{\de h_i} 
		&= 1 - \eta m_i ^ {1/\nu} \ \frac{\de }{\de h_i} \rho_i ^ {-1/\nu} 
			\nonumber \\
		&= 1 + \eta m_i ^ {1/\nu}  \frac{1}{\nu} \ \rho_i ^ {-1/\nu - 1} \frac{\de \rho_i}{\de h_i} 
			\nonumber \\
		&= 1 + \eta m_i ^ {1/\nu}  \frac{1}{\nu} \ \rho_i ^ {-1/\nu - 1} \sum_{n} m_j \frac{\de}{\de h_i} W(|\R_i - \R_j|, h_i) 
			\nonumber \\
		&= 1 - \eta m_i ^ {1/\nu}  \frac{1}{\nu} \ \rho_i ^ {-1/\nu - 1} \sum_{n} m_j \frac{1}{h_i} \left( \nu W(\x_i) + r_{ij}  \DELDR{W(\x_i)} \right)
			\nonumber \\
		&= 1 - \frac{\eta}{h_i}  \frac{m_i ^ {1/\nu}}{\rho_i^{1 + 1/\nu}}  \sum_{n} m_j \left( W(\x_i) + \frac{r_{ij}}{\nu}   \DELDR{W(\x_i)} \right)
\end{align}


However, a perhaps better version can be obtained via

\begin{align}
 	h_i 
 		&= \eta n^{-\nu} \nonumber \\
 	\text{Then} 
 	\quad n h_i^\nu 
 		&= \eta^\nu \nonumber \\
	\text{Giving us } 
	\quad f 
		&= f(h_i) = n h_i^\nu - \eta^\nu  = h_i^\nu  \sum_{j} W(|\R_i - \R_j|, h_i) - \eta^\nu = 0\\
	\frac{\de f}{\de h_i} 
		&=  \frac{\de }{\de h_i} \left( h_i^{\nu} \sum_{j} W(|\R_i - \R_j|, h_i) \right)
			\nonumber \\
		&= \nu h_i^{\nu - 1} \sum_{j} W(|\R_i - \R_j|, h_i) + h_i^\nu \sum_{j} \frac{\de }{\de h_i} W(|\R_i - \R_j|, h_i)
			\nonumber \\
		&= \nu h_i^{\nu - 1}\sum_{j} W(|\R_i - \R_j|, h_i) - h_i^\nu \sum_{j}\frac{1}{h_i} \left( \nu W(\x_i) + r_{ij}  \DELDR{W(\x_i)}\right)
\end{align}


The main advantage is that instead of the mass density, only the number density $n = \oldsum_{j} W(|\R_i - \R_j|, h_i)$ is used, and therefore the determination of the smoothing length (and with it the determination of the resolution) only depends on particle positions and no other fluid properties.
This doesn't change much for simulations where all SPH particles have equal masses, but in finite volume meshless methods, where masses are exchanged via fluxes, the smoothing lengths will be weighted by the variable particle masses.
This is also the implemented version.







%==================================================================================
\subsection{$\eta$ and $N_{neigh}$}
%==================================================================================

Add discussion that is in \cite{priceSmoothedParticleHydrodynamics2012a}.






