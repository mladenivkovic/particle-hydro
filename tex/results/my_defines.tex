%==============================================
% This file contains my definitions and
% newcommands. Makes things easy to copypaste
% between projects.
%==============================================


%--------------------------------------------
% Math Stuff
%--------------------------------------------

\newcommand{\corresponds}{\mathrel{\widehat{=}}}       % equals with hat

\newcommand {\arctanh}{\mathrm{arctanh}}               % Atanh
\newcommand{\arccot}{\mathrm{arccot }}                 % Acotanh

\newcommand{\limz}[1]{\lim\limits_{#1 \rightarrow 0}}  % Limes of something towards zero

\newcommand{\bm}{\boldmath}                            % Bold font in math
\newcommand{\dps}{\displaystyle}                                               

\newcommand{\e}{\mbox{e}}                              % e noncursive in math mode

\newcommand{\del}{\partial}                            % partial diff operator
\newcommand{\de}{\mathrm{d}}                           % differential d
\newcommand{\D}{\mathrm{d}}                            % differential d
\newcommand{\GRAD}{\mathrm{grad}\ }                    % gradient
\newcommand{\DIV}{\mathrm{div}\ }                      % divergence
\newcommand{\ROT}{\mathrm{rot}\ }                      % rotation

\newcommand{\CONST}{\mathrm{const.\ }}                 % constant
\newcommand{\var}{\mathrm{var}}                        % variance

\newcommand{\g}{^\circ}                                % degrees
\newcommand{\degr}{^\circ}                             % degrees

\newcommand{\msol}{M_\odot}                            % solar mass
\newcommand{\order}{\mathcal{O}}                       % order, e.g. O(h^2)


\newcommand{\x}{\mathbf{x}}                            % x vector
\newcommand{\xdot}{\dot{\mathbf{x}}}                   % x dot vector
\newcommand{\xddot}{\ddot{\mathbf{x}}}                 % x doubledot vector
\newcommand{\R}{\mathbf{r}}                            % r vector
\newcommand{\rdot}{\dot{\mathbf{r}}}                   % r dot vector
\newcommand{\rddot}{\ddot{\mathbf{r}}}                 % r doubledot vector
\newcommand{\vel}{\mathbf{v}}                          % v vector
\newcommand{\V}{\mathbf{v}}                            % v vector
\newcommand{\vdot}{\dot{\mathbf{v}}}                   % v dot vector
\newcommand{\vddot}{\ddot{\mathbf{v}}}                 % v doubledot vector

\newcommand{\dete}{\mathrm{d}t}                        % dt
\newcommand{\delte}{\del t}                            % partial t
\newcommand{\dex}{\mathrm{d}x}                         % dx
\newcommand{\delx}{\del x}                             % partial x
\newcommand{\der}{\mathrm{d}r}                         % dr
\newcommand{\delr}{\del r}                             % partial r


\newcommand{\deldx}{\frac{\del}{\del x}}				% shortcut partial derivative, in line
\newcommand{\ddx}{\frac{\de}{\de x}}					% shortcut total derivative, in line
\newcommand{\DELDX}[1]{\frac{\del  #1}{\del x}}			% shortcut partial derivative, on fraction
\newcommand{\DDX}[1]{\frac{\de  #1}{\de x}}				% shortcut total derivative, on fraction

\newcommand{\deldvecx}{\frac{\del}{\del \x}}	   		% shortcut partial derivative, in line
\newcommand{\ddvecx}{\frac{\de}{\de \x}}				% shortcut total derivative, in line
\newcommand{\DELDVECX}[1]{\frac{\del  #1}{\del \x}}		% shortcut partial derivative, on fraction
\newcommand{\DDVECX}[1]{\frac{\de  #1}{\de \x}}			% shortcut total derivative, on fraction

\newcommand{\deldr}{\frac{\del}{\del r}}				% shortcut partial derivative, in line
\newcommand{\ddr}{\frac{\de}{\de r}}					% shortcut total derivative, in line
\newcommand{\DELDR}[1]{\frac{\del  #1}{\del r}}			% shortcut partial derivative, on fraction
\newcommand{\DDR}[1]{\frac{\de  #1}{\de r}}				% shortcut total derivative, on fraction







%-----------------------------------------------
% Work related / project specific math stuff
%-----------------------------------------------

\newcommand{\Aij}{$\mathbf{A}_{ij}$}	% A_ij
\newcommand{\Aijm}{\mathbf{A}_{ij}}		% A_ij math
\newcommand{\U}{\mathbf{U}}				% State vector
\newcommand{\F}{\mathbf{F}}				% Flux tensor
\newcommand{\psitilde}{\tilde{\psi}}	% psi tilde
\newcommand{\half}{1/2}









%----------------------------------
% Redefinitions
%----------------------------------


% replace \sum with \sum\limits
\let\oldsum\sum
\renewcommand{\sum}{\oldsum\limits}






%-----------------------------------
% Shortcuts
%-----------------------------------

% quickly insert a figure without a caption
% 		usage: \quickfig{filename}{label}
\newcommand{\quickfig}[2]{
       \begin{figure}[H]
               \includegraphics[width=\textwidth]{#1}
               \caption{\label{#2}}
       \end{figure}
}


\newcommand{\quickfigcap}[3]{
       \begin{figure}[H]
               \includegraphics[width=\textwidth]{#1}
               \caption{#3\label{#2}}
       \end{figure}
}



 \usepackage[framemethod=TikZ]{mdframed}

% New Colors (needs to be after usepackage mdframed)
 \definecolor{babyblueeyes}{rgb}{0.63, 0.79, 0.95}
 \definecolor{ashgrey}{rgb}{0.7, 0.75, 0.71}
 \definecolor{caribbeangreen}{rgb}{0.0, 0.8, 0.6}
 \definecolor{bittersweet}{rgb}{1.0, 0.44, 0.37}

 \mdfdefinestyle{todo}{%
     %rightline=true,
     innerleftmargin=10,
     innerrightmargin=10,
     %frametitlerule=true,
     %frametitlerulecolor=black,
     frametitlebackgroundcolor=bittersweet,
     frametitlerulewidth=2
 }

\newcommand{\todo}[1]{\begin{mdframed}[style=todo,frametitle={TODO}] #1 \end{mdframed}}
