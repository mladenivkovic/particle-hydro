\newpage
%======================================================
\section{Notation}
%======================================================



\todo{Needs revision}

We are working on numerical methods.
Both space and time will be discretized.

Space will be discretized in cells which will have integer indices to describe their position.
Time will be discretized in fixed time steps, which may have variable lengths.
Nevertheless the time progresses step by step.

The lower left corner has indices $(0, 0)$ in 2D.
In 1D, index 0 also represents the leftmost cell.




We have:
\begin{itemize}
	\item integer subscript: Value of a quantity at the cell, i.e. the center of the cell. Example: $\U_i$, $\U_{i-2}$ or $\U_{i, j+1}$ for 2D.
	\item non-integer subscript: Value at the cell faces, e.g. $\F_{i-\half}$ is the flux at the interface between cell $i$ and $i-1$, i.e. the left cell as seen from cell $i$.
	\item integer superscript: Indication of the time step. E.g. $\U ^ n$: State at timestep $n$
	\item non-integer superscript: (Estimated) value of a quantity in between timesteps. E.g. $\F^{n + \half}$: The flux at the middle of the time between steps $n$ and $n + 1$.
\end{itemize}